%%%%%%%%%%%%%%%%%%%%%%%%%%%%%%%%%%%%%%%%%%%%%%%%%%%%%%%%%%%%%%%%%%%%%%%
%% Spezielle Kommandos für die Bearbeitung des Projekts
% Fügt eine leere Seite ein
\newcommand*{\blankpage}{
   \clearpage{
      \pagestyle{empty}
      \cleardoublepage
   }
}

% Ändert die Schriftart
\newcommand*{\changefont}[3]{
\fontfamily{#1}  \fontseries{#2}  \fontshape{#3}  \selectfont}

% Verhalten der Gleichungs-Zähler
\newcommand*{\subeq}{\renewcommand\theequation{\theparentequation{}-\arabic{equation}}}

%%%%%%%%%%%%%%%%%%%%%%%%%%%%%%%%%%%%%%%%%%%%%%%%%%%%%%%%%%%%%%%%%%%%%%%
%% Kommandos zum Einfügen von Querverweisen
%% Deutsche Version

% Markiert eine indirekte Zitierung.
% #1: Seitennummer
% #2: Literaturreferenz
\newcommand*{\vgl}[2]{(vgl. \cite[S.~#2]{#1})}

% Markiert eine indirekte Zitierung mit 2 Verweisen
% #1: Seitennummer Verweis 1
% #2: Literaturreferenz 1
% #3: Seitennummer Verweis 2
% #4: Literaturreferenz 2
\newcommand*{\Vgl}[4]{(vgl. \cite[S.~#2]{#1} und \cite[S.~#4]{#3})}

% Deklariert eine Quelle.
% #1: Literaturreferenz
\newcommand*{\source}[1]{Quelle:~\cite{#1}}

% Verweist auf eine Quelle
% #1: Literaturreferenz
\newcommand*{\refSource}[1]{\cite{#1}}

% Verweist auf zwei Quellen
% #1: Literaturreferenz 1
% #2: Literaturreferenz 2
\newcommand*{\RefSource}[2]{\cite{#1, #2}}

% Fügt einen Verweis zu einem Bild ein.
% #1: Referenzname
\newcommand*{\refFig}[1]{\textbf{Abbildung~\ref{#1}}}

% Fügt einen Verweis zu einer Tabelle ein.
% #1: Referenzname
\newcommand*{\refTab}[1]{Tabelle~\ref{#1}}

% Fügt einen Verweis zu einem Kapitel ein.
% #1: Referenzname
\newcommand*{\refChpt}[1]{\textbf{Kapitel~\ref{#1}}}

% Fügt einen Verweis zu einem Abschnitt ein.
% #1: Referenzname
\newcommand*{\refSec}[1]{\textbf{Abschnitt~\ref{#1}}}
\newcommand*{\refAtt}[1]{\textbf{Anhang~\ref{#1}}}

% Fügt einen Verweis zu einem Unterabschnitt ein.
% #1: Referenzname
\newcommand*{\refSubSec}[1]{Unterabschnitt~\ref{#1}}

% Fügt einen Verweis zu einem Codeabschnitt ein.
% #1: Referenzname
\newcommand*{\refCode}[1]{Quellcode~\ref{#1}}

% Fügt einen Verweis zu einer Gleichung ein.
% #1: Referenzname
\newcommand*{\refEqu}[1]{Gleichung~\textnormal{(\ref{#1})}}

% Fügt ein kursiven Querverweis zu einem Bild ein.
% #1: Referenzname
\newcommand*{\seeFig}[1]{(s. \refFig{#1})}

% Fügt ein kursiven Querverweis zu einer Tabelle ein.
% #1: Referenzname
\newcommand*{\seeTab}[1]{(siehe \refTab{#1})}

% Fügt ein kursiven Querverweis zu einem Abschnitt ein.
% #1: Referenzname
\newcommand*{\seeSec}[1]{(s. \refSec{#1})}
\newcommand*{\seeAtt}[1]{(s. \refAtt{#1})}

% Fügt ein kursiven Querverweis zu einem Kapitel ein.
% #1: Referenzname
\newcommand*{\seeChpt}[1]{(siehe \refChpt{#1})}

% Fügt ein kursiven Querverweis zu einem Codeabschnitt ein.
% #1: Referenzname
\newcommand*{\seeCode}[1]{(siehe \refCode{#1})}

% Fügt ein kursiven Querverweis zu einer Gleichung ein.
% #1: Referenzname
\newcommand*{\seeEqu}[1]{(siehe \refEqu{#1})}

% Fügt ein kursiven Querverweis zu zwei Gleichungen ein.
% #1: Referenzname 1
% #2: Referenzname 2
\newcommand*{\SeeEqu}[2]{(siehe \refEqu{#1} und \refEqu{#2})}

\newcommand{\addtotoc}[1]{
\cleardoublepage
\phantomsection
\addcontentsline{toc}{chapter}{\textbf{#1}}
}

%%%%%%%%%%%%%%%%%%%%%%%%%%%%%%%%%%%%%%%%%%%%%%%%%%%%%%%%%%%%%%%%%%%%%%%
\makeatletter
\newcommand{\faculty}[1]{\gdef\@faculty{#1}}
\newcommand{\@faculty}{\@latex@warning@no@line{No \noexpand\faculty given}}
\newcommand{\thesisType}[1]{\gdef\@thesisType{#1}}
\newcommand{\@thesisType}{\@latex@warning@no@line{No \noexpand\thesisType given}}
\newcommand{\erstpruefer}[1]{\gdef\@erstpruefer{#1}}
\newcommand{\@erstpruefer}{\@latex@warning@no@line{No \noexpand\erstpruefer given}}
\newcommand{\zweitpruefer}[1]{\gdef\@zweitpruefer{#1}}
\newcommand{\@zweitpruefer}{\@latex@warning@no@line{No \noexpand\zweitpruefer given}}
\newcommand{\matrikelnummer}[1]{\gdef\@matrikelnummer{#1}}
\newcommand{\@matrikelnummer}{\@latex@warning@no@line{No \noexpand\matrikelnummer given}}

\renewcommand*{\maketitle}{
\begin{titlepage}
   \centering
   \vspace*{\fill}
   {\textbf{\Huge{\@title} \label{title}} \par}
   \vspace{1cm}
   {\LARGE \@subject \par}
   \vspace{2.3cm}
   {\LARGE{{\@thesisType} von \par \@author} \par}
   {\vspace{2.6cm}}
   {\LARGE{\@date}}
   \vfill
\end{titlepage}
}
\makeatother
