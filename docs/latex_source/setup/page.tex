% Dokumentation des KOMA-Script-Packets: scrbook
%%%%%%%%%%%%%%%%%%%%%%%%%%%%%%%%%%%%%%%%%%%%%%%%%%%%%%%%%%%%%%%%%%%%%%%
%% Optionen zum Layout des Artikels                                  %%
%%%%%%%%%%%%%%%%%%%%%%%%%%%%%%%%%%%%%%%%%%%%%%%%%%%%%%%%%%%%%%%%%%%%%%%
\documentclass[            %% KOMA-Skript Dokumentklassen verwenden
draft = false,             		% Entwurfsmodus
paper = A4,                		% Papierart
% paper = landscape,       		% Seite im Querformat setzen
pagesize = pdftex,         		% Pagesize an pdfTeX anpassen
% pagesitze = xetex,       		% Pagesize an XeTeX anpassen
fontsize = 12pt,           		% Schriftgröße (12pt, 11pt (Standard))
%BCOR = 10mm,                	% Bindekorrektur
DIV=15,                    		% Satzspiegelfaktor, je größer, desto mehr Text auf Seite
twoside = false,           		% Doppelseiten (Ein|Aus|Semi)
twocolumn = false,         		% zweispaltiger Satz
parskip = full,           		% Absatzformatierung s. scrguide 3.1
%abstract = true,         		% Überschrift über Abstract an (nicht bei scrbook aktiv)
chapterprefix = false,      		% Layout der Kapitelüberschriften
appendixprefix = true,     		% Layout der Anhangüberschriften
headinclude = false,       		% Kopfzeile zu Text betrachten (für Satzspiegelberechnung)
footinclude = false,       		% Fußzeile zu Text betrachten (für Satzspiegelberechnung)
mpinclude = false,         		% Rand zum Text betrachten (für Satzspiegelberechnung)
% headlines = 1.25,        		% Anzahl der Kopfzeilen
% headheight = 2cm,        		% Höhe der Kopfzeile
% headsepline = true,      		% Trennline zum Seitenkopf
% footsepline = true,      		% Trennline zum Seitenfuß
numbers = auto,            		% KOMA-Skript setzt Endpunktierung im Inahltsverzeichnis.
cleardoublepage = plain,   		% Einstellung des Seitenstils für eingefügte Vakatseiten
% leqno,                   		% Nummerierung von Gleichungen links
% fleqn,                   		% Ausgabe von Gleichungen linksbündig
footnotes = multiple,      		% Mehrer aufeinander folgende Fußnoten durch Komma trennen
titlepage = true,          		% Titelei auf eigener Seite \maketitle[Seitenanzahl]
%headings = openright,     		% Kapitel auf rechter Seite beginnen, ggf. Vakatseiten
headings = normal,         		% Normalgroße Überschriften
open = right,              		% Seiten rechts beginnen
%toc = listof,
%toc = listof,              		% Abb.- und Tab.verzeichnis im Inhaltsverzeichnis
%toc = noindex,             		% Index im Inhaltsverzeichnis (index|noindex)
%toc = sectionentrywithoutdots,
%toc = chapterentrywithoutdots,
%bibliography = openstyle,  		% Offener Darstellungsstil der Bibliographie
listof = chaptergapline,   		% Abb.- und Tab.verzeichnis werden nach Kapitel geordnet
overfullrule = true,
numbers=noenddot                % keine Punkte in Kapitelnummern
]{scrbook}
