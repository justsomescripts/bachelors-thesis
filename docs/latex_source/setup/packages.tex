\pagestyle{empty}
% Deutsche Anpassungen %%%%%%%%%%%%%%%%%%%%%%%%%%%%%%%%%%%%%%%%%%%%%%%%%%%%%
\usepackage[T1]{fontenc}				% Schriftkodierung
\usepackage[utf8]{inputenc}				% Eingabekodierung

\usepackage[english,ngerman]{babel}		% Neue deutsche Rechtschreibung verwenden
\addto\captionsngerman{					% Name des Literaturverzeichnisses verändern
  \renewcommand{\bibname}{Quellenverzeichnis}
}
\usepackage{microtype}					% verbessert Umbruch von Texten
\usepackage{scrhack}

% URL %%%%%%%%%%%%%%%%%%%%%%%%%%%%%%%%%%%%%%%%%%%%%%%%%%%%%%%%%%%%%%%%%%%%%%
\usepackage{url}						% Links durch \url{}

\usepackage[
  left=3.8cm,
  right=2.8cm,
  top=3.5cm,
  bottom=3.5cm,
]{geometry}

\usepackage[normalem]{ulem}

\makeatletter
\let\oldTitle\title
\renewcommand{\title}[1]{
   \oldTitle{#1}
   \gdef\@utitle{\uline{#1}}
}
\newcommand{\@utitle}{\@latex@warning@no@line{No \noexpand\Title given}}
\newcommand{\underlinedtitle}{\@utitle}
\makeatother

\usepackage[sfdefault]{roboto}					% Roboto Schriftart
\usepackage[section]{placeins}

\usepackage[                  % Package für Farbe
   pdftex,                    % Ausgabe für pdfTeX
   %xetex,                    % Ausgabe für XeTeX
   hyperref,                  % Hyperref-Support
   svgnames,                             % Zusätzliche Farben per Name hinzufügen
   x11names,                             % un nochmehr Farben
   showerror                  % Fehlermeldung bei Verwendung undefinierter Farbe
   ]{xcolor}%

\definecolor{dkgreen}{rgb}{0,0.6,0}
\definecolor{mauve}{rgb}{0.58,0,0.82}

%% Packages für Grafiken & Abbildungen %%%%%%%%%%%%%%%%%%%%%%%%%%%%%%%%%%%%%%
\usepackage[pdftex]{graphicx}			% Zum Laden von Grafiken
   \graphicspath{{Bilder/}}				% Standardpfad für Bilder/Grafiken

\usepackage{fancybox}		% Ermöglicht zeichnen von Boxen von definierter Länge


% fancy cross-referencing \fref & \Fref
\usepackage[german, plain]{fancyref}


\usepackage[                  % Bildunterschriften
   format = hang,             % Ausrichtung
%   indention = 0.5cm,        % Texteinzug ab 2. Zeile
   labelformat = default,     % Bezeichnerformatierung
   labelsep = colon,          % Bezeichner-Trennzeichen
%  textformat = period,       % Textformatierung (Textendzeichen)
   justification = justified, % Text als Blocksatz setzen
   singlelinecheck = true,    % 1-Zeilen-Beschriftungen --> zentrieren erlaubt
   font = {it, small},        % Beschriftungsformat
   labelfont = bf,			  % Bezeichnerformat (Bild x.y...)
%   textfont = it,            % Textformat
   margin = 0pt,              % Beschriftungsrand (auch links/rechts möglich {0pt, 0pt})
   oneside,                   % Einseitiger Textsatz
%   twoside,                  % Zweiseitiger Textsatz
   parskip = 5pt,             % Abstand zwischen Absätzen in Beschriftungen
   skip = 8pt,                % Abstand von Beschriftung
   listfigurename = Bilderverzeichnis,	% Abbildungen anstatt Abbildungsverzeichniss
   listtablename  = Tabellen,           % Tabellen anstatt Tabellenverzeichnis
   figurewithin = chapter,    % Zählerbegrenzung festlegen (also Bild 3.1 anstatt Bild 43)
   tablewithin = chapter      % Zahlerbegrenzung festlegen
   ]{caption}

% Tabellen %%%%%%%%%%%%%%%%%%%%%%%%%%%%%%%%%%%%%%%%%%%%%%%%%%%%%%%%%%%%%%%%%%%%%%%%%
\usepackage{array}             % Standarderweiterung für Tabellen
\usepackage{diagbox}		   % Diagonale linie durch Zelle

% Bibliographiestil %%%%%%%%%%%%%%%%%%%%%%%%%%%%%%%%%%%%%%%%%%%%%%%%%%%%%%%%%%%%%%%%
%\usepackage[				% Erweiterung der Basisfunktionalität
%	numbers,				% Nummern
%	square,					% Eckige Klammern
%    sort&compress,			% Multiple Zitate sortieren und zusammenfassen
%    longnamesfirst			% Erstes Zitat mit vollen Namen
%   ]
%   {natbib}

%{Bibliographie/ka-style}
%   \setcitestyle{
%     authoryear,% Author-Year-Style
%      square,   % Eckige Klammern
%     semicolon, % semicolon als Trennzeichen
%      aysep{},% Kein explizites Trennzeichen zwischen Author und Year
%      yysep{;}% Komma als Trennzeichen zwischen mehreren Jahren
%      }

% Listings %%
\usepackage{listings}
\usepackage{lipsum}

% Spaltendefinition rechtsbndig mit definierter Breite
%\newcolumntype{w}[1]{>{\raggedleft\hspace{0pt}}p{#1}}

%% Mathematik-Packages%%%%%%%%%%%%%%%%%%%%%%%%%%%%%%%%%%%%%%%%%%%%%%%%%
\usepackage{amsmath}             % AMS Mathematikumgebungen/-erweiterungen
\usepackage{amssymb}             % AMS Symbol/Font-Paket
%\usepackage{accents}            % Erweiterung für Akzente im Mathematik-Modus
\usepackage{mathcomp}            % Symbolerweiterungen für Mathematikmodus
%\usepackage{cancel}             % Durchstreicehn in Mathematikumgebung

% Title format Anpassen
\makeatletter
\renewcommand{\chapterlinesformat}[3]{\@hangfrom{\bfseries{#2}}\mdseries{#3}}
\renewcommand{\sectionlinesformat}[4]{\@hangfrom{\hskip#2\bfseries{#3}}{\mdseries{#4}}}
\makeatother

% TODO: handle more than only chapter and section

\usepackage[                  % Paket zum Navigieren innerhalb eines Dokuments
   plainpages=true,           % Arabische Zeichen für Link-Darstellung
   bookmarksopen=true,        % Lesezeichenbaum aufklappen
   bookmarksopenlevel=1,      % Aufklapptiefe
   pdfborder={0 0 0},         % Rahmenfarbe
   pdfpagemode=UseOutlines,	% Ansicht im Adobe Reader
   pdftex,                    % Ausgabe mit pdfTeX
   pdfusetitle
   ]{hyperref}
   \hypersetup{
   pdfborder={0 0 0},			% Kein Rand um Links
	plainpages=false,			% Zur korrekten Erstellung der Bookmarks
	hypertexnames=false			% Zur korrekten Erstellung der Bookmarks
   }


\usepackage[
   printonlyused
   ]{acronym}

\usepackage{upgreek}

\lstset{ %
    basicstyle=\footnotesize\ttfamily,	% the size of the fonts that are used for the code
    numbers=left,						% where to put the line-numbers
    numbersep=5pt,						% how far the line-numbers are from the code
    backgroundcolor=\color{white},      % choose the background color. You must add \usepackage{color}
    showspaces=false,               % show spaces adding particular underscores
    showstringspaces=false,         % underline spaces within strings
    showtabs=false,                 % show tabs within strings adding particular underscores
    rulecolor=\color{black},        % if not set, the frame-color may be changed on line-breaks within not-black text (e.g. commens (green here))
    tabsize=2,                      % sets default tabsize to 2 spaces
    captionpos=t,                   % sets the caption-position to bottom
    breaklines=true,                % sets automatic line breaking
    breakatwhitespace=false,        % sets if automatic breaks should only happen at whitespace
    title=\lstname,                 % show the filename of files included with \lstinputlisting;
									% also try caption instead of title
    keywordstyle=\color{blue},      % keyword style
    commentstyle=\color{dkgreen},   % comment style
    escapeinside={\%*}{*)},         % if you want to add LaTeX within your code
    morekeywords={*,...}            % if you want to add more keywords to the set
    frame=b,
    stringstyle=\color{mauve}\ttfamily,	% Farbe der String
    showspaces=false,					% Leerzeichen anzeigen ?
    showtabs=false,						% Tabs anzeigen ?
    xleftmargin=17pt,
    framexleftmargin=17pt,
    framexrightmargin=5pt,
    framexbottommargin=4pt,
    showstringspaces=false      % Leerzeichen in Strings anzeigen ?
}

\DeclareCaptionFont{white}{\color{white}}
\DeclareCaptionFormat{listing}{\colorbox[cmyk]{0.43, 0.35, 0.35,0.01}{\parbox{\textwidth}{\hspace{15pt}#1#2#3}}}
%\captionsetup[lstlisting]{format=listing,labelfont=white,textfont=white, singlelinecheck=false, margin=0pt, font={bf,footnotesize}}

\usepackage{tabularx}
\usepackage{booktabs}
\usepackage{caption}
\usepackage{csquotes}
\usepackage{helvet}
\usepackage{mathpazo}
\usepackage[binary-units=true]{siunitx}
\usepackage{subcaption}
\usepackage{enumitem}
\setlist[description]{style=nextline}
\usepackage{romannum}

\renewcaptionname{ngerman}\figureautorefname{Abb.}

\tolerance 1414
\hbadness 1414
\emergencystretch 1.5em
\hfuzz 0.3pt
\widowpenalty=10000
\vfuzz \hfuzz
\raggedbottom

% XXX: ugly
%\makeatletter
%  \let\ps@plain\ps@empty
%\makeatother

\DeclareTOCStyleEntries[linefill=\hfill]{tocline}{section,subsection,subsubsection,figure,table}
\DeclareTOCEntryStyle{section}{section}
\addtokomafont{chapterentry}{\mdseries}
\addtokomafont{chapterentrypagenumber}{\bfseries}
\newcommand*{\SavedOriginaladdchaptertocentry}{}
\let\SavedOriginaladdchaptertocentry\addchaptertocentry

\setuptoc{toc}{indent}
\RedeclareSectionCommand[tocpagenumberformat=\bfseries]{section}%
\RedeclareSectionCommand[tocpagenumberformat=\bfseries]{subsection}%

\renewcommand*{\addchaptertocentry}[2]{%
   \SavedOriginaladdchaptertocentry{\bfseries{#1}}{
       \texorpdfstring{\textbf{#2}}{#2}}
}

\newcommand*{\SavedOriginaladdsectiontocentry}{}
\let\SavedOriginaladdsectiontocentry\addsectiontocentry
\renewcommand*{\addsectiontocentry}[2]{%
   \SavedOriginaladdsectiontocentry{\bfseries{#1}}{#2}
}

\newcommand*{\SavedOriginaladdsubsectiontocentry}{}
\let\SavedOriginaladdsubsectiontocentry\addsubsectiontocentry
\renewcommand*{\addsubsectiontocentry}[2]{%
   \SavedOriginaladdsubsectiontocentry{\bfseries{#1}}{#2}
}

\makeatletter
%\renewcommand\frontmatter{%
%   \cleardoublepage
%   \@mainmatterfalse
%   \pagenumbering{Roman}}
%\makeatother