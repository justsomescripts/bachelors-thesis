\chapter{Einleitung}

Roboter spielen aufgrund der schnellen Anpassungsmöglichkeit bei Produktionsprozessen, die heute einen hohen Stellenwert hat, eine immer wichtigere Rolle in der Industrie. Mit steigender Rechenleistung und dem schnellen Fortschritt im Bereich der Bilderkennung, vor allem unter Anwendung von \textit{\acp{CNN}}, findet die Erkennung von Objekten mit Kamerasystemen in immer mehr Bereichen der Robotik Anwendung. So können beispielsweise schon heute auf \textit{\ac{DL}} basierende Techniken, die noch vor Kurzem nur selten für einen praktischen Einsatz geeignet waren, Anwendung in der Industrie finden. Zukünftig kann \textit{\ac{DL}} für die Vereinfachung der Bedienbarkeit von Robotern und eine weiter reichende Automatisierung von Prozessen genutzt werden \cite{diesing_how_2021}. Dabei erfolgt die Programmierung von Robotersteuerungen meist mithilfe herstellerspezifischer Programme, die an bestimmte Hardware gebunden sind. Diese Herstellerbindung bringt Nachteile bei der Interaktionsfähigkeit mehrerer Roboter mit sich. Auch die Integration von Komponenten wie Kameras und weiteren Sensoren, die nicht vom jeweiligen Hersteller der Robotersteuerung unterstützt werden, erschwert die Automatisierung von Prozessen und erhöht die Umsetzungskosten. Diese Problematik versucht das Open-Source-Projekt \ac{ROS} zu lösen. Das seit 2007 kontinuierlich weiterentwickelte System verfolgt das Ziel, die Steuerung verschiedener Roboter zu standardisieren und eine gemeinsame Schnittstelle zu schaffen \cite[Kapitel~1]{quigley_ros_2009}. Der öffentlich zugängliche Quellcode bietet die Möglichkeit, eigene Anpassungen vorzunehmen und zur Entwicklung beizutragen. Neben der Interoperabilität der Programmteile stellen aufgrund des Open-Source-Codes auch die umfangreiche Verfügbarkeit anwendungsspezifischer Pakete und Anwendungen, die mit der vom Hersteller bereitgestellten Schnittstelle nicht realisierbar sind, große Vorteile des \ac{ROS} dar. Durch die Möglichkeit, eigene mit \ac{ROS} kompatible Programme zu veröffentlichen, erfolgt eine schnelle und kontinuierliche Weiterentwicklung des Systems.

Da \ac {ROS} Industrial nicht nur in der klassischen Robotik eingesetzt wird, sondern auch in anderen Bereichen wie beispielsweise der Forschung an autonomen Fahrzeugen, ergibt sich ein breites Spektrum an Softwarepaketen mit unterschiedlichen Anwendungszwecken. Dazu zählt unter anderem das Programm \textit{darknet\_ros}, das eine Schnittstelle zwischen \textit{\ac{YOLO}}, einem Algorithmus zur Objekterkennung, und \ac{ROS} Industrial zur Verfügung stellt. Die Erkennung, Unterscheidung und Ortung von Objekten ist, wenn überhaupt eine Schnittstelle vorhanden ist, bei Nutzung von Systemen der Hersteller oft hardwaregebunden und nur schwierig umsetzbar. 

Das Ziel dieser Arbeit ist die Programmierung einer Robotersteuerung unter Anwendung des \ac{ROS} Industrial und eines 3D-Kamerasystems. Diese soll aus einer ungeordneten Zusammenstellung von Bauteilen festgelegte Teile erkennen und gemäß einem variablen Bauplan anordnen. Die Bauteile sollen hierbei, wenn bestimmte Bedingungen wie eine passende Größe und Oberflächenbeschaffenheit eingehalten werden, nach einem Einlernprozess frei gewählt werden können. Das Programm soll außerdem mit wenigen Anpassungen unabhängig von der Kameraposition am Endeffektor und der Roboterkinematik funktionieren. Es kann als Grundlage für viele Anwendungen, die zum Beispiel eine einfach anpassbare Sortierung von Objekten oder die Unterscheidung und Lokalisierung ungeordneter Teile erfordern, genutzt werden.

Wie kann ein solches System zur Erkennung von Art und Position ungeordneter Bauteile unter Verwendung des Robot Operating System umgesetzt werden?

Zur Beantwortung dieser Frage befasst sich die vorliegende Arbeit primär mit der Objekterkennung. Die Implementierung eines Bauplans und der Ansteuerung wird in einer separaten Arbeit behandelt \cite[Dennis Steinbeck: Entwicklung einer KI-basierten Robotersteuerung mit Industrial ROS - Generierung von Handhabungssequenzen für mechanische Konstruktionen nach gegebenen Bauplänen]{steinbeck_entwicklung_2022}. Für die Umsetzung wird aufgrund der Verfügbarkeit relevanter Pakete und der Stabilität und Zuverlässigkeit die Version \textit{\ac{ROS}-I Noetic} genutzt. Die Objekterkennung erfolgt mit einem Intel RealSense 435 3D-Kamerasystem, das am Endeffektor des 6-achsigen Knickarmroboters RV6L von Reis Robotics befestigt ist. Der Hauptteil der Arbeit ist in drei Teile gegliedert. Der erste Teil befasst sich mit den technischen Grundlagen, die zum Verständnis des Gesamtsystems notwendig sind oder das gewählte Vorgehen begründen. Anschließend wird auf die verwendete Software und Hardware zur Objekterkennung und Steuerung des Roboters eingegangen. Der dritte Teil bietet eine Übersicht über die Zusammenhänge des Gesamtsystems und den Ablauf von Erkennung bis zur Handhabung der Objekte.