\chapter{Fazit}

Die Nutzung von 3D-Kamerasystemen und \textit{\acp{CNN}} in Verbindung mit Robotern schafft eine gute Grundlage für die Handhabung verschiedener Teile. An den Ergebnissen werden der Fortschritt und die Möglichkeiten aktueller neuronaler Netzte wie \textit{\ac{YOLO}}, die auch mit vergleichsweise schwacher Hardware geboten werden, deutlich. Durch die Nutzung des \ac{ROS} ist eine standardisierte Schnittstelle für die Kommunikation verschiedener Pakete verfügbar. Hier erschließen sich die Vorteile der Open-Source-Software besonders. Die Verwendung bestehender Pakete als Grundlage für die Implementierung der Objekterkennung unter Nutzung des Meta-Betriebssystems schafft eine gute Basis für die Umsetzung der Bilderkennung. 

Das Ziel der Erkennung von Klasse und Position einer ungeordneten Zusammenstellung von Objekten wurde erreicht. Das auf \textit{darknet\_ros\_3d} basierende \textit{rv6l\_3d} kann mithilfe des \ac{ROS} und eines 3D-Kamerasystems zuverlässig einzelne Bauteile im Aufnahmebereich erkennen und Positionsdaten und Art an den \textit{Subscriber} des Handhabungsprogramms weitergeben. Die Intel RealSense 435 eignet sich durch die kleine Baugröße, genaue Abstandsberechnung und für den Anwendungsfall ausreichende Auflösung der Kameras gut für die durch \textit{rv6l\_3d} umgesetzte Positionserkennung. Ein Nachteil der genutzten Konfiguration ist die Nähe zu einer singulären Stellung in der Scanposition. Eine Befestigung des Kamerasystems parallel zum Endeffektor kann diesen lösen, was allerdings eine Verringerung des Erkennungsbereichs zur Folge hat. Die Nutzung von \textit{\ac{DL}} zur Erkennung der Objekte bietet im Vergleich zu traditioneller \textit{\ac{CV}} viele Vorteile, wie die hohe Erkennungsrate oder den vergleichsweise wenig komplexen Prozess zur Erstellung eigener Datensätze. So eignet sich \textit{\ac{YOLO}} in den Grundzügen wegen der für ein \textit{\ac{CNN}} guten Performance und kontinuierlich zuverlässigen Erkennung der verwendeten Objekte sehr gut für den Anwendungsfall.

Auch wenn eine Grundfunktionalität gegeben ist, muss das \textit{rv6l\_3d}-Paket im Hinblick auf die in \refSec{sec:evaluierung} beschriebenen Probleme optimiert werden. Neben notwendigen Verbesserungen wie der Weitergabe mehrerer Bauteile durch die Schnittstelle zwischen Objekterkennungs- und Handhabungsprogramm und der Ausnahmebehandlung können die erwähnten Vorschläge wie die Positionserkennung zu einer Erweiterung des Anwendungsbereichs genutzt werden.

Die Ergebnisse zeigen die durch \ac{ROS} gebotenen Möglichkeiten zur Kombination verschiedener Hardware und Software. Dies wird besonders in der Übersicht des Gesamtsystems \seeFig{fig:gesamtsystem_struktur} deutlich. Durch diese Eigenschaft bietet das System eine Basis, die für zukünftige Weiterentwicklungen genutzt werden kann.